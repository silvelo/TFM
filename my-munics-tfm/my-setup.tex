% !TEX root = my-tfm.tex

% **************************************************
% Files' Character Encoding
% **************************************************
%% Not necessary with luaLaTeX
% 
\PassOptionsToPackage{utf8}{inputenc}
\usepackage{inputenc}


% **************************************************
% Information and Commands for Reuse
% **************************************************
\newcommand{\thesisTitle}{Sistema de autenticación continua basado en el comportamiento mediante técnicas de streaming}
\newcommand{\thesisName}{Arturo Silvelo Pallín}
\newcommand{\thesisSubject}{Trabajo Fin de Máster}
\newcommand{\thesisDate}{Curso 2020/2021}

\newcommand{\thesisFirstSupervisor}{José Carlos Dafonte Vázquez}
\newcommand{\thesisSecondSupervisor}{Daniel Garabato Míguez}

\newcommand{\thesisUniversityStudies}{\protect{Máster Inter-Universitario en Ciberseguridad}}
\newcommand{\thesisUniversity}{Universidad de A Coruña}     % Replace with your university
\newcommand{\thesisUniversitySchool}{Facultad de Informática} % Replace with your school
\newcommand{\thesisUniversityCity}{A Coruña}  % Replace with your city
\newcommand{\thesisUniversityStreetAddress}{Campus de Elviña}
\newcommand{\thesisUniversityPostalCode}{15071}


% **************************************************
% Debug LaTeX Information
% **************************************************
%\listfiles

% **************************************************
% Load and Configure Packages
% **************************************************
%\usepackage[spanish]{babel} % babel system, adjust the language of the content

\usepackage{polyglossia}
\setdefaultlanguage{spanish}

\PassOptionsToPackage{% setup clean thesis style
    figuresep=colon,
    hangfigurecaption=false,
    hangsection=true,
    hangsubsection=true,
    sansserif=false,
    configurelistings=true,
    colorize=full,
    colortheme=bluemagenta,
    configurebiblatex=true,
    bibsys=biber,
    bibfile=bib-refs,
    bibstyle=numeric-comp, %numeric-comp, numeric-verb, alphabetic 
    bibsorting=nty,
}{munics}
\usepackage{munics}

\hypersetup{% setup the hyperref-package options
    pdftitle={\thesisTitle},    %   - title (PDF meta)
    pdfsubject={\thesisSubject},%   - subject (PDF meta)
    pdfauthor={\thesisName},    %   - author (PDF meta)
    plainpages=false,           %   -
    colorlinks=false,           %   - colorize links?
    pdfborder={0 0 0},          %   -
    breaklinks=true,            %   - allow line break inside links
    bookmarksnumbered=true,     %
    bookmarksopen=true          %
}

% **************************************************
% Other Packages
% **************************************************
%\usepackage{fontspec}
\usepackage{scrhack}
\usepackage{subcaption}
\usepackage[justification=centering]{caption}
\usepackage{multirow}
\usepackage{booktabs}
\usepackage{tablefootnote}
\usepackage{amsmath}
\usepackage{makecell}
\usepackage{tabularx}
\usepackage{xltabular}
\usepackage{ragged2e}
\usepackage[official]{eurosym}
\usepackage{amsmath}
\usepackage{longtable}
\usepackage[capitalize]{cleveref}
\usepackage{xcolor}

% **************************************************
% Caption Setup
% **************************************************
\captionsetup{belowskip=12pt,aboveskip=4pt}

% **************************************************
% Tabular X
% **************************************************
\newcolumntype{C}{>{\Centering\arraybackslash}X} % centered "X" column
\newcolumntype{S}{>{\hsize=.4\hsize}X} % Small column

% **************************************************
% Formulas
% **************************************************
\DeclareMathOperator{\atantwo}{atan_2}
\makeatletter
\DeclareRobustCommand{\atan}{%
    \operatorname{atan}%
    \@ifnextchar2{_}{}%
}
\makeatother


% **************************************************
% Custom Figure
% **************************************************
\newcommand{\cfig}[4][.9\linewidth]{
    \begin{figure}[hbtp!]
        \centering
        \includegraphics[width=#1, keepaspectratio]{#2}
        \caption{#3}
        \label{#4}
    \end{figure}
}