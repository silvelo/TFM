% **************************************************
% Abstract
% **************************************************
\pdfbookmark[0]{Abstract}{Abstract}
\addchap*{Abstract}
\label{sec:abstract}

Nowadays, authentication systems allow to verify the user's identities at a specific point in the system, usually at login, while leaving the user with the responsibility of logging off each time they stop using their computer. Because of this step, in multiuser and professional environments, users often leave their sessions open so that other users can access and perform unauthorized operations on their behalf. These actions can result in access to confidential data that could cause legal problems for both the organization and the legitimate user, that may result in financial penalties.

To avoid these problems, new authentication systems have been developed, such as fingerprint schemes, but this entails a cost in price due to the installation of a reader for each workstation. Additionally, operating systems can implement restriction policies that close inactive sessions and force users to change their passwords in short periods of time. However, such measures may disturb the user and lead to the use of weak passwords, causing a false sense of security.


For this purpose, an authentication system has been proposed in this project as a second factor mechanism, in which the user's identities are periodically verified according to their behavior during their sessions. Thus, data related to mouse movement is gathered in order to build unique user profiles by means of Artificial Intelligence techniques that were combined with streaming processing methods to achieve a real-time authentication process. As a result, a scalable system able of authenticating users in a transparent manner to them was successfully developed and deployed.



{\vspace{5mm}\textbf{\textit{Keywords ---}}
    Mouse dynamics,
    Behavioral biometrics,
    Security monitoring
}


% **************************************************
% Resumen
% **************************************************
\pdfbookmark[0]{Resumen}{Resumen}
\addchap*{Resumen}
\label{sec:resumen}

Los sistemas de autenticación actuales permiten verificar la identidad de los usuarios en un momento concreto del sistema, por lo general al inicio, lo que deja al usuario la responsabilidad de cerrar la sesión cada vez que abandone su ordenador. Debido a este paso, en entornos multiusuario y en el ámbito profesional los usuarios suelen dejar sus sesiones abiertas permitiendo que otros usuarios accedan y realicen operaciones en su sesión. Estas acciones pueden derivar en acceso a datos confidenciales que provocarían problemas legales tanto a la organización como al usuario legítimo, resultando en posibles sanciones económicas.

Para evitar estos problemas se han implementado nuevos sistemas de autenticación, como los esquemas basados en la huella dactilar, pero esto conlleva un coste en precio debido a la necesidad de instalar un lector por cada puesto de trabajo. Por otra parte, los sistemas operativos permiten configurar políticas de restricción que cierran sesiones inactivas y obligan a los usuarios a cambiar sus contraseñas en cortos periodos de tiempo. Sin embargo, estas medidas pueden provocar malestar en los usuarios y llevar al uso de contraseñas débiles, lo que acaba provocando una falsa sensación de seguridad.


Para este propósito, en este proyecto se ha planteado un sistema de autenticación como segundo factor, en el que se verifica la identidad del usuario periódicamente a partir de su comportamiento en el uso del dispositivo. Para ello, se recopilan los datos relativos al movimiento del ratón para la creación de perfiles de usuario únicos con técnicas de Inteligencia Artificial que combinado con procesamiento \textit{streaming} realice el proceso de autenticación en tiempo real.  De esta manera, se ha implementado y desplegado un sistema escalable capaz de autenticar a los usuarios de manera transparente para ellos.


{\vspace{5mm}\textbf{\textit{Palabras clave ---}}
Movimientos de ratón,
Biometría del comportamiento,
Monitorización de seguridad
}
