\chapter{Trabajo Futuro}
\label{sec:future_work}

En este capítulo se comentarán algunos de los posibles desarrollos que se podrían llevar a cabo en un futuro.

\begin{itemize}
    \item \textbf{Refinar el proceso de extracción y selección de características:} Partiendo de la implementación actual, se propone explorar nuevas características en la búsqueda de una optimización de los resultados.
    
    \item \textbf{Actualización de los modelos:} Este tipo de sistemas suelen estar entrenados con un conjunto de datos iniciales que es persistente en el tiempo, pero el comportamiento de una persona puede cambiar con el paso del tiempo, debido a circunstancias como enfermedades o lesiones. Por lo tanto, un aprendizaje que evolucionase con el usuario permitiría obtener una mayor fiabilidad al adaptarse a cambios en los hábitos y comportamiento de los usuarios.    

    \item \textbf{Explorar nuevas técnicas:} Las técnicas elegidas a lo largo del proyecto son una buena base de referencia para el proyecto, pero creemos que diseños basados en tecnologías de \textit{deep-learning} pueden mejorar los resultados obtenidos.
    

    \item \textbf{Combinar técnicas:} Las técnicas usadas se han utilizado de forma independiente, una futura rama de investigación sería combinar las técnicas y obtener de esta formar una predicción por votación.
    
    \item \textbf{Nuevas agrupaciones:} En el proyecto se emplea una agrupación de 1000 eventos con el fin de optimizar el envío y procesamiento de datos. Se plantea realizar un estudio más detallado sobre el tamaño de bloque, buscando de esta forma encontrar un equilibrio entre la mejora de resultados y la degradación en los tiempos de procesamiento.
    
    \item \textbf{Mejorar la web de administración:} La aplicación web permite mostrar la información del estado actual del sistema para realizar las acciones que se consideren oportunas. Este tipo de acciones como un mensaje de aviso o cerrar la sesión del usuario podrían ser implementadas para que se realicen desde la propia aplicación.
\end{itemize}
